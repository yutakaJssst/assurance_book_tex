\chapter{モデルベースシステムズエンジニアリング}
%高井さん
\label{chap4}

\section{システムエンジニアリングの基礎}

\subsection{システムエンジニアリングとは}

システムエンジニアリングは、INCOSE SE HANDBOOKによると、「システムの実現を成功させることができる複数の専門分野にまたがるアプローチおよび手段」と定義されています。この分野は以下の役割を果たします:

\begin{itemize}
    \item システム開発に必要な概念の提供
    \item システム開発で必要な活動の提供
\end{itemize}

システムエンジニアリングは、分野に依存せず「システム」を構築したり発展させたりするための知見をまとめたものです。これは、民間の巨大船舶、軍事システム、スマホアプリ/業務システム、ミッションを達成するためのシステム、人系のシステムなど、多岐にわたる分野に適用可能です。

\subsection{システムエンジニアリングの知見}

システムエンジニアリングの知見には主に以下の3つが含まれます:

\begin{enumerate}
    \item システムに関する概念の定義
    \item システムの開発や運用においてありうる活動の定義
    \item システムを記述するために必要な観点の提供
\end{enumerate}

\subsection{システムライフサイクルプロセス群}

システムエンジニアリングでは、システム開発に関わる活動を分類し、標準化したものとしてシステムライフサイクルプロセス群が定義されています。ISO/IEC/IEEE 15288:2015によると、これらのプロセス群は以下の4つに分類されます:

\begin{itemize}
    \item 合意プロセス群
    \item 組織のプロジェクトイネーブリングプロセス群
    \item テクニカルマネジメントプロセス群
    \item テクニカルプロセス群
\end{itemize}

これらのプロセス群は、システムの直接的な開発活動だけでなく、それを支援する活動も含んでいます。

\section{モデルベースドシステムズエンジニアリング(MBSE)}

\subsection{MBSEの概念}

モデルベースドシステムズエンジニアリング(MBSE)は、モデルを活用したシステムエンジニアリングのアプローチです。ここでいうモデルとは、ある対象に対して、その対象の特徴を理解したり予測したりするために用いられる抽象的な表現を指します。

\subsection{SysML(Systems Modeling Language)}

SysMLは、システムエンジニアリングのための代表的なモデリング言語です。UML(Unified Modeling Language)をベースに定義されており、システムを以下の四つの側面から記述することが可能です:

\begin{enumerate}
    \item 構造
    \item 振る舞い
    \item パラメータ間の関係
    \item 要求
\end{enumerate}

\section{SysMLを用いたシステムモデリング}

\subsection{構造のモデリング}

構造のモデリングでは、システムを構成する要素とその関係を表現します。例えば、自動車システムの場合、エンジン、シャーシ、車体などの構成要素とそれらの関係を表現します。

\subsection{振る舞いのモデリング}

振る舞いのモデリングでは、システムの動作や状態遷移を表現します。例えば、自動車の運転プロセスや、エンジンの始動から停止までの状態遷移などを表現します。

\subsection{パラメータ間の関係のモデリング}

パラメータ間の関係のモデリングでは、システムの性能や特性を決定する要因間の関係を表現します。例えば、自動車の場合、エンジン出力と最高速度の関係、車体重量と燃費の関係などを表現します。

\subsection{要求のモデリング}

要求のモデリングでは、システムが満たすべき条件や制約を表現します。例えば、自動車の場合、安全性基準、燃費基準、排出ガス規制などの要求を表現します。

\section{トレードオフ分析}

\subsection{トレードオフ分析の概要}

トレードオフ分析は、システムの異なる特性や性能間の関係を評価し、最適な解決策を見出すプロセスです。多くの場合、ある特性を向上させると他の特性が低下するという関係があり、これらのバランスを取ることが重要です。

\subsection{記述型モデルと分析型モデルの連携}

トレードオフ分析を効果的に行うためには、記述型モデルと分析型モデルの連携が重要です。

\begin{itemize}
    \item 記述型モデル:SysMLなどを用いて、システムの構造、振る舞い、要求などを記述します。ステークホルダー間のコミュニケーションや合意形成に役立ちます。
    \item 分析型モデル:MATLAB/Simulinkなどのツールを用いて、システムの性能や動作をシミュレーションします。定量的な評価や予測に役立ちます。
\end{itemize}

\subsection{トレードオフ分析の手順}

以下の手順でトレードオフ分析を行います:

\begin{enumerate}
    \item ソリューション案の検討と妥当性確認
    \item 分析ケースの検討と妥当性確認
    \item パラメータ項目/値の検討と妥当性確認
    \item シミュレーションなどによる分析の実施
    \item 分析結果の評価と意思決定
\end{enumerate}

\section{事例研究: 自動駐車システム}

ここでは、自動駐車システムの設計と評価を例に、システムエンジニアリングの実践を見ていきます。

\subsection{システム概要}

想定する自動駐車システムは以下の特徴を持ちます:

\begin{itemize}
    \item ショッピングセンター近くの巨大な立体駐車場を対象
    \item 利用者は駐車場入口でクルマから降り、アプリで自動駐車を指示
    \item 車両は自律的に空きスペースまで移動し駐車
    \item 引き取り時も、アプリで指示後に車両が自動で出庫
\end{itemize}

\subsection{能力の定義と評価指標}

システムに求められる能力とその評価指標を定義します:

\begin{itemize}
    \item 能力:自動駐車機能
    \item 評価指標:
    \begin{itemize}
        \item 平均入庫時間
        \item 平均出庫時間
        \item 平均待ち時間
        \item 安全度レベル
        \item 利用者ストレスポイント
    \end{itemize}
\end{itemize}

\subsection{リソースアーキテクチャの検討}

システムを構成するリソースとその構造を検討します。主要なコンポーネントには以下が含まれます:

\begin{itemize}
    \item 自動運転機能付き車両
    \item 駐車場管理システム
    \item 空きスペース確認システム(固定カメラまたはドローン)
    \item ユーザーインターフェース(スマートフォンアプリ)
\end{itemize}

\subsection{分析ケースの定義}

システムの評価のための分析ケースを定義します。考慮すべき要素には以下が含まれます:

\begin{itemize}
    \item 自動運転車普及率(例:20%、80%)
    \item 来客状況(閑散期、繁忙期)
    \item 駐車場の構造と容量
    \item 周辺交通状況
\end{itemize}

\subsection{シミュレーションと分析}

定義した分析ケースに基づき、MATLAB/Simulinkなどのツールを用いてシミュレーションを行います。また、STAMP/STPAなどの手法を用いて安全性分析を実施します。

\subsection{結果の評価とトレードオフ分析}

シミュレーションと分析の結果を評価し、各ソリューション案のトレードオフを検討します。例えば:

\begin{itemize}
    \item 固定カメラ方式:初期コストは低いが、スケーラビリティに課題
    \item ドローン方式:柔軟性が高いが、安全管理に追加コストが必要
\end{itemize}

これらの結果を総合的に判断し、最適なソリューションを選択します。

\section{まとめ}

システムエンジニアリングは、複雑なシステムの開発と管理を体系的に行うための重要なアプローチです。MBSEやSysMLの活用、そしてトレードオフ分析の実施により、以下のような利点が得られます:

\begin{itemize}
    \item システムの全体像と詳細の両方を把握できる
    \item ステークホルダー間のコミュニケーションが促進される
    \item 定量的な評価に基づく意思決定が可能になる
    \item システムの品質、信頼性、安全性の向上につながる
\end{itemize}

今後のシステム開発者は、これらの手法と考え方を効果的に活用し、より良いシステムを設計・開発することが求められます。

