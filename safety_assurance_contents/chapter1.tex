\chapter{イントロダクション}
\label{chap1}

\section{システムのディペンダビリティ:基礎概念と現代的課題}

\subsection{システムとは何か}

私たちの日常生活において、「システム」という言葉をよく耳にします。しかし、この概念を正確に定義するのは簡単ではありません。ここでは、参考文献[x]の定義を示します。

システム: 他のシステムと相互作用するもの。ここでいうシステムは、与えられた環境であり、ハードウェア、ソフトウェア、人間、および自然現象を伴う物質的世界などである。システム境界はシステムとその環境が接する境目である。

システムは互いにサービスを介して相互作用を行います。参考文献[x]のサービスの定義を示します。

サービス:システムが(提供者の役割によって)提供するサービスとは、その利用者から見える振る舞いである。ここで利用者とは、提供者からサービスを受ける別のシステムである。提供者のシステム境界のうちサービス提供がなされる部分を提供者のサービスインタフェースという。

システムは常に特定の環境に置かれており、その環境との相互作用を通じて機能します。例えば、自動車というシステムは道路という環境の中で機能し、気象条件や交通状況などの環境要因の影響を受けます。

上記の定義は抽象的ですが、サービスは、システムによって私たちに提供される機能や便益のことです。私たちユーザーは、このサービスを通じてシステムと関係を持ちます。例えば、スマートフォンというシステムは、通話、メッセージング、インターネット閲覧などのサービスを提供し、私たちはこれらのサービスを通じてスマートフォンと関わっています。

\subsection{ディペンダビリティの概念}

ディペンダビリティ(Dependability)は、直訳すると「依存可能性」となりますが、システム工学では「総合信頼性」と訳されることが多い重要な概念です。この概念の起源は、あるシステムが別のシステムに依存(Depend)することができる、そのシステムの性質を表すことにあります。

例えば、運転手(システムA)が自動車(システムB)に依存する場合を考えてみましょう。自動車がディペンダブル(依存可能)であるためには、どのような性質を持つべきでしょうか?安全であること、運転しやすいこと、目的地まで確実に到達できることなどが挙げられるでしょう。このように、ディペンダビリティは利用者や環境によって求められる性質が異なる、相対的な概念です。

システムがディペンダブルであるためには、以下の条件を満たす必要があります:

\begin{itemize}
\item 利用者や環境において望まれる性質を持ち続けること
\item サービスを継続的に提供すること
\end{itemize}

さらに、システムがその性質を失った場合(つまり、ディペンダブルでなくなった場合)には、速やかに復旧を行い、サービスを継続する能力も求められます。

\section{ディペンダビリティの体系と用語}

ディペンダビリティの概念は、1980年代から国際的な研究グループ(IFIP WG 10.4 "Dependable Computing and Fault Tolerance"など)によって体系化され、用語の整理が行われてきました。当初は「耐故障性(Fault Tolerance)」研究から発展し、近年ではセキュリティの概念も含めて議論されています。

ここでは、Avizienis et al. (2004)による体系に基づいて、ディペンダビリティの主要な概念を紹介します。

\subsection{ディペンダビリティ属性}

ディペンダビリティ属性は、システムがディペンダブルであるために持つべき特性を表します。主な属性には以下のものがあります:

\begin{itemize}
\item 可用性(Availability):正しいサービスの即応性
\item 信頼性(Reliability):正しいサービスの継続性
\item 安全性(Safety):利用者と環境へ破壊的影響をもたらさないこと
\item 一貫性(Integrity):不適切なシステム変更がないこと
\item 保守性(Maintainability):変更と修理を受け入れられること
\end{itemize}

これらの属性は相互に関連しており、時には相反する関係にあることもあります。システム設計者は、対象システムの要求に応じてこれらの属性のバランスを取る必要があります。

\subsection{ディペンダビリティへの脅威}

システムのディペンダビリティを脅かす要因は、以下の3つの概念で整理されています:

\begin{itemize}
\item 欠陥(Fault):エラーの原因となるとみなされる、あるいは推定されるもの、こと
\item 誤り(Error):障害が起こりうるシステムの状態(ただし、エラー状態になったからといって、必ずしも障害が起こるとは限らない)
\item 障害(Failure):サービスが正しいサービスから逸脱する出来事
\end{itemize}

これらの概念は因果関係にあり、欠陥がエラーを引き起こし、エラーが障害につながる可能性があります。

\subsection{ディペンダビリティへの脅威に対処する手段}

ディペンダビリティへの脅威に対処するため、以下の4つの手段が提案されています:
\begin{itemize}
\item 欠陥防止(Fault Prevention):欠陥の導入や発生を防ぐ
\item 耐故障性(Fault Tolerance):欠陥がある中で障害を防ぐ
\item 欠陥除去(Fault Removal):欠陥の数や深刻度を減らす
\item 欠陥予測(Fault Forecasting):欠陥の現在の数、今後の障害、影響などを予測する
\end{itemize}
これらの手段を適切に組み合わせることで、システムのディペンダビリティを向上させることができます。

\section{現代のシステムとディペンダビリティの課題}

\subsection{情報システムの規模拡大とネットワーク化}

近年、情報システムは急速に規模を拡大し、ネットワーク化が進んでいます。この変化は以下のような段階を経ています:

\begin{itemize}
\item 単機能システム
\item 複合機能システム
\item ネットワーク化されたシステム
\item サービスポータル化されたシステム
\end{itemize}

この変化に伴い、ITシステムは生活・社会インフラとしての役割を担うようになり、組込みシステムもポータル化が進んでいます。これにより、システムの複雑さと重要性が増大し、ディペンダビリティの確保がより困難かつ重要になっています。

\subsection{大規模なシステム障害のリスク}

システムの大規模化と複雑化に伴い、以下のような要因により大規模なシステム障害のリスクが高まっています:
\begin{itemize}
\item プログラムサイズの増大と多機能化
\item ブラックボックス化したコンポーネントの増加
\item 技術進化のスピードの加速
\item 接続システムの多様化
\item 利害関係者の変化と要求の頻繁な変更
\end{itemize}

これらの要因により、システムの完全な理解と制御が困難になっています。

\subsection{オープンシステム(開放系)の問題}

現代のシステムは、複雑化、ネットワーク化し、そのためそのシステムのステークホルダーの誰もが
\begin{itemize}
\item 仕様/実装の不完全さ:要求、仕様、設計、実装、テストの各段階での不完全さが避けられない
\item システムの完全理解の困難さ:構成要素の論理的不透明さ(複雑化、巨大化、ブラックボックス化)により、システム全体の挙動予測が困難
\item 使用環境の変化に伴う不確実さ:要求事項・レベルの変化、想定外の使われ方、ネットワークを介しての構成要素の変化
\item セキュリティリスクの増大:ネットワークを介した外部からの意図的な攻撃のリスク
\end{itemize}

\section{これからのディペンダビリティに向けて}

\subsection{不完全・不確実なシステムへの対応}

従来の形式手法やテストなどの手法に加え、不完全・不確実なシステムがディペンダブルであるための新たなアプローチが必要とされています。しかし、完全に障害を排除することは不可能であり、深刻な障害が起こる可能性は以前よりも高まっています。

\subsection{システム保証(System Assurance)の重要性}

このような状況下で重要になってくるのが「システム保証」の概念です。システム保証とは、システムがどの程度ディペンダブルか、あるいはディペンダブルでないか、リスク分析などをもとに利用者などのステークホルダーに説明し、納得してもらうプロセスです。

絶対に安全である、あるいは完全にディペンダブルであることは不可能であるという事実を、ステークホルダーに理解してもらうことが重要です。

\subsection{説明責任の必要性の高まり}

近年、システムの安全性や信頼性に関する説明責任の重要性が高まっています。例えば、2009-2010年のトヨタ プリウスの北米大規模リコール問題では、原因説明への準備不足が指摘されました。

また、自動車の機能安全規格であるISO 26262の制定により、自動車メーカーはより説明しやすい形で安全性の根拠を示すことが求められるようになりました。

\subsection{AI技術とディペンダビリティ}

最近では、AI(人工知能)技術を用いたシステム、特に自動運転システムなどのディペンダビリティが重要な課題となっています。AI技術の不確実性や説明可能性の問題は、従来のシステムとは異なる新たなディペンダビリティの課題を提起しています。


システムのディペンダビリティは、もはや単なる技術的な問題ではなく、社会的、倫理的な問題としても捉えられるようになっています。システムの複雑化、不確実性の増大、そしてAI技術の台頭により、従来のディペンダビリティの概念や手法だけでは不十分になってきています。

これからのディペンダビリティ確保には、技術的な対策に加えて、システム保証と説明責任の遂行が不可欠です。また、不完全性や不確実性を前提としたシステム設計と運用の考え方を確立していく必要があります。

システム開発者、運用者、そして利用者を含むすべてのステークホルダーが、これらの課題を理解し、協力してディペンダブルなシステムの実現に取り組むことが求められています。

\section*{参考文献}
\begin{enumerate}
\item 
\end{enumerate}

