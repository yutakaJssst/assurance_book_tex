\chapter{アシュアランスケース}
\label{chap3}

人工知能の研究開発は加速的に進み始めている。2022年に登場したChatGPTは、誰でも簡単にウエブ上で質問をすることができるチャットボットであるが、その回答の詳細さと自然さに、多くの人が驚いた。また歩行者や標識を自動認識する人工知能を持つ自動運転車は、アメリカのWaymo社や日本のTier4社など、多くの企業が開発競争を繰り広げており、アメリカではすでにカリフォルニア州において自動運転タクシーが実用化されている。しかしながら、チャットボットや自動認識を行う人工知能は、100\%正しい出力をするわけではない。であるにも関わらず、その圧倒的な利便性から、人工知能が組み込まれたシステムが加速的に普及していくことはより確実になってきている。そのような状況おいて、我々を取り巻くシステムが安心して利用できるものなのか、改めて社会的な合意形成が必要な時期になっている。

\section{アシュアランスケースの記述方法}
アシュアランスケースは基本的にドキュメントであり、多くの場合文書形式で記述される。近年では、アシュアランスケースの主張と議論構造、およびエビデンスのつながりをモデル化したグラフィカルな記述方法も用いられている。本著では、グラフィカルな記述方法の中で、Goal Structuring Notaion (GSN)を用いる。

GSNは主に以下のノードの種類からなる。
\begin{itemize}
\item ゴール (Goal) : 
システムに対して、議論すべき命題である。例えば「システムはディペンダブルである」とか「システムは適切な安全性をみたす」などである。
\item 戦略(Strategy): ゴールが満たされることを、サブゴールに分割して詳細化するときの議論の仕方である。例えば、「システムは安全である」というゴールに対して、現時点で識別されているハザードに対処できていることによって議論したいとき、戦略ノードとして「識別されたハザードごとに場合分け」を用いると、例えばひとつのサブゴールは「システムはハザードXに対処できる」となる。
\item 前提(Context): ゴールや戦略を議論するとき、その前提となる情報である。例えば、運用環境や、システムのスコープ、あるいは「識別されたハザードのリスト」などである。
\item 未達成(Undeveloped): ゴールを保証するための十分な議論もしくはエビデンスがないことを表す。
\end{itemize}
\section{アシュアランスケースの例}
\section{アシュアランスケース記述ステップ}

\section{基本演習}
